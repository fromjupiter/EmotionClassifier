\documentclass{article} % For LaTeX2e
\usepackage{nips15submit_e,times}
\usepackage{hyperref}
\usepackage{url}
%\documentstyle[nips14submit_09,times,art10]{article} % For LaTeX 2.09


\title{Face Emotion Classification Using Single-Layer Neural Networks}


\author{
Kexiang Feng \\
Department of Computer Science and Engineering\\
University of California, San Diego\\
\texttt{fkxcole@gmail.com} \\
\And
Xupeng Yu \\
Department of Electrical and Computer Engineering \\
University of California, San Diego \\
\texttt{maiyuxiaoge@gmail.com}
}

% The \author macro works with any number of authors. There are two commands
% used to separate the names and addresses of multiple authors: \And and \AND.
%
% Using \And between authors leaves it to \LaTeX{} to determine where to break
% the lines. Using \AND forces a linebreak at that point. So, if \LaTeX{}
% puts 3 of 4 authors names on the first line, and the last on the second
% line, try using \AND instead of \And before the third author name.

\newcommand{\fix}{\marginpar{FIX}}
\newcommand{\new}{\marginpar{NEW}}

\nipsfinalcopy % Uncomment for camera-ready version

\begin{document}


\maketitle

\begin{abstract}
This paper summarizes steps to build a face emotion neural network without using any machine learning libraries(Sklearn, Pytorch, etc.). We start with a logistic regression layer to solve binary classification problem, then we use a softmax layer to deal with more general multi-label classification. All models are evaluated on the CK+ dataset using 10-fold cross validation. Our logistic regressor achieves an average accuracy of $ 100\% $ in Happy vs Angry classification and our softmax regressor achieves an average accuracy of 78\% in 6-way classification.
\end{abstract}

\section{Introduction}
In what section we describe what problems to get the scores on what part of the assignment.

In section 2 we briefly introduce our data preprocessing and data split, namely Principal Component Analysis and K-Fold. Section 3 and 4 each elaborates our work in the logistic regressor and softmax regressor. Following that we share our findings in Section 5. Lastly we briefly introduce individual contributions in section 6.

\section{Dataset and Task Description}

what is CK+? what is aligned vs resized? What preprocess (PCA) do we need?
DATA\_SAMPLER

\subsection {Principal Component Analysis}
\subsection {K-Fold Cross Validation}


\section{LOGISTIC REGRESSION}
Basic ideas of LR. 
\subsection {Evaluation on Happiness vs Anger using the resized dataset}
\subsection {Evaluation on Happiness vs Anger using the aligned dataset}
\subsection {Evaluation on Fear vs Surprise using the aligned dataset}


\section{SOFTMAX REGRESSION}
Basic idea of softmax regression
\subsection {Weight Visualization}
\subsection {Evaluation}
\subsection {Batch vs Stochastic Gradient Descent}
\subsection {Class weights to handle imbalanced dataset}

\section {CONCLUSION}
\section{INDIVIDUAL CONTRIBUTIONS}
Our group consists of two members: \textit{Kexiang Feng} and \textit{Xupeng Yu}. \\
\textit{Kexiang Feng} builds the PCA, K-Fold modules and Softmax regressor. He also wrote the scripts to visualize the softmax weights and PCA components. \\
\textit{Xupeng Yu} builds the Logistic regressor. 

\section {References}
\small{
[1]
}

\end{document}
